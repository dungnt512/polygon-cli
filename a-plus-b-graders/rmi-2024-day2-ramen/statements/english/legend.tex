Có $N$ sinh viên $S_0, S_1, \ldots, S_{N - 1}$ và $N$ môn học tự chọn $M_0, M_1, \ldots, M_{N - 1}$. Mỗi sinh viên $S_i$ có một mức độ yêu thích $A_{ij}$ đối với từng môn học $M_j$. Mức độ yêu thích càng cao thì sinh viên càng muốn học môn đó. Với mỗi sinh viên, các mức độ yêu thích với các môn học là phân biệt --- tức là không có hai môn học nào được đánh giá bằng nhau.

Do điều kiện tổ chức, trong một đợt đăng ký, mỗi môn học chỉ có một suất duy nhất và mỗi sinh viên chỉ được học đúng một môn.

Giả sử sinh viên đăng ký môn học theo một thứ tự cho trước $\pi = (\pi_0, \pi_1, \dots, \pi_{N-1})$, tức là sinh viên $S_{\pi_0}$ đăng ký trước, tiếp đến $S_{\pi_1}$, v.v. Mỗi sinh viên, khi đến lượt mình, sẽ chọn môn học mà mình yêu thích nhất trong số các môn còn lại (chưa có ai đăng ký).

Sau khi tất cả sinh viên đã đăng ký, ta xác định được mỗi sinh viên $S_i$ được gán một môn học $M_{\sigma(i)}$, và tổng mức độ hài lòng của cả nhóm là:

\begin{center}
$$
\sum_{i = 0}^{N - 1} A_{i, \sigma(i)}
$$
\end{center}

\textbf{Mục tiêu:} Hãy tìm thứ tự đăng ký $\pi$ sao cho tổng mức độ hài lòng là lớn nhất.

\Interaction
Bạn cần viết hàm sau:
\begin{center}
\begin{lstlisting}
vector<int> solve(int N);
\end{lstlisting}
\end{center}

Trong đó $N$ là số sinh viên (cũng là số môn học). Hàm trả về một hoán vị $\pi$ của các chỉ số từ $0$ đến $N - 1$, đại diện cho thứ tự các sinh viên đăng ký môn học.

Bạn được phép gọi tối đa 750 lần hàm:
\begin{center}
\begin{lstlisting}
vector<pair<int, int>> ask(const vector<int>& order);
\end{lstlisting}
\end{center}
Trong đó:
\begin{itemize}
    \item Tham số \texttt{order} là thứ tự đăng ký các sinh viên.
    \item Kết quả trả về là một danh sách gồm $N$ cặp $(\sigma(i), A_{i, \sigma(i)})$, trong đó $\sigma(i)$ là môn học mà sinh viên $S_i$ nhận được, và $A_{i, \sigma(i)}$ là mức độ yêu thích tương ứng.
\end{itemize}